% Options for packages loaded elsewhere
\PassOptionsToPackage{unicode}{hyperref}
\PassOptionsToPackage{hyphens}{url}
%
\documentclass[
  11pt,
]{article}
\usepackage{amsmath,amssymb}
\usepackage{iftex}
\ifPDFTeX
  \usepackage[T1]{fontenc}
  \usepackage[utf8]{inputenc}
  \usepackage{textcomp} % provide euro and other symbols
\else % if luatex or xetex
  \usepackage{unicode-math} % this also loads fontspec
  \defaultfontfeatures{Scale=MatchLowercase}
  \defaultfontfeatures[\rmfamily]{Ligatures=TeX,Scale=1}
\fi
\usepackage{lmodern}
\ifPDFTeX\else
  % xetex/luatex font selection
\fi
% Use upquote if available, for straight quotes in verbatim environments
\IfFileExists{upquote.sty}{\usepackage{upquote}}{}
\IfFileExists{microtype.sty}{% use microtype if available
  \usepackage[]{microtype}
  \UseMicrotypeSet[protrusion]{basicmath} % disable protrusion for tt fonts
}{}
\makeatletter
\@ifundefined{KOMAClassName}{% if non-KOMA class
  \IfFileExists{parskip.sty}{%
    \usepackage{parskip}
  }{% else
    \setlength{\parindent}{0pt}
    \setlength{\parskip}{6pt plus 2pt minus 1pt}}
}{% if KOMA class
  \KOMAoptions{parskip=half}}
\makeatother
\usepackage{xcolor}
\usepackage[left = 2.5cm, right = 2cm, top = 2cm, bottom =
2cm]{geometry}
\usepackage{longtable,booktabs,array}
\usepackage{calc} % for calculating minipage widths
% Correct order of tables after \paragraph or \subparagraph
\usepackage{etoolbox}
\makeatletter
\patchcmd\longtable{\par}{\if@noskipsec\mbox{}\fi\par}{}{}
\makeatother
% Allow footnotes in longtable head/foot
\IfFileExists{footnotehyper.sty}{\usepackage{footnotehyper}}{\usepackage{footnote}}
\makesavenoteenv{longtable}
\usepackage{graphicx}
\makeatletter
\def\maxwidth{\ifdim\Gin@nat@width>\linewidth\linewidth\else\Gin@nat@width\fi}
\def\maxheight{\ifdim\Gin@nat@height>\textheight\textheight\else\Gin@nat@height\fi}
\makeatother
% Scale images if necessary, so that they will not overflow the page
% margins by default, and it is still possible to overwrite the defaults
% using explicit options in \includegraphics[width, height, ...]{}
\setkeys{Gin}{width=\maxwidth,height=\maxheight,keepaspectratio}
% Set default figure placement to htbp
\makeatletter
\def\fps@figure{htbp}
\makeatother
\setlength{\emergencystretch}{3em} % prevent overfull lines
\providecommand{\tightlist}{%
  \setlength{\itemsep}{0pt}\setlength{\parskip}{0pt}}
\setcounter{secnumdepth}{5}
\usepackage{float}
\usepackage{sectsty}
\usepackage{paralist}
\usepackage{setspace}\spacing{1.0}
\usepackage{fancyhdr}
\usepackage{lastpage}
\usepackage{dcolumn}
\usepackage{natbib}\bibliographystyle{agsm}
\usepackage[nottoc, numbib]{tocbibind}
\ifLuaTeX
  \usepackage{selnolig}  % disable illegal ligatures
\fi
\usepackage{bookmark}
\IfFileExists{xurl.sty}{\usepackage{xurl}}{} % add URL line breaks if available
\urlstyle{same}
\hypersetup{
  hidelinks,
  pdfcreator={LaTeX via pandoc}}

\author{}
\date{\vspace{-2.5em}}

\begin{document}

\subsectionfont{\raggedright}
\subsubsectionfont{\raggedright}

\pagenumbering{gobble}

\begin{centering}

\vspace{3cm}


\begin{center}\includegraphics[width=0.8\linewidth,height=0.8\textwidth]{../../../Users/jaimem/University of Tasmania/IMAS - Documents/Approved Logos/UniTas_IMAS_P_Pos_Col_RGB_2021} \end{center}

\vspace{4cm}

\LARGE

\doublespacing
{\bf Estimating Recreational Abalone Harvest Using Length-Weight Data from Commercial Catch Sampling}

\vspace{4 cm}

\Large
{\bf Institute for Marine and Antarctic Studies}

\Large
{\bf Univeristy of Tasmania}

\vspace{1cm}

\normalsize
\singlespacing
By

\vspace{0.5 cm}

\large

{\bf Jaime McAllister and Craig Mundy}

\vspace{1.5 cm}

\normalsize
June 2022

\end{centering}

\newpage

\section{Background}\label{background}

Effort in the recreational abalone fishery is managed by daily limits,
and until the 2019/2020 recreational fishing season the daily bag limit
was set at a maximum of 10 abalone/licence/day, with a possession limit
of 20 at any point in time. From 2010 to 2019 persistent declines in the
commercial abalone fishery catch and catch rate were observed despite
multiple decreases in the commercial TACC. Commercial CPUE is used as
the sole proxy for biomass of the abalone resource, on the basis that
the recreational catch is both small and difficult to accurately
quantify. In 2019, a recommendation for closure of five key areas of the
Tasmanian East Coast commercial blacklip abalone fishery was accepted.
Similarly a reduction (but not closure) of the recreational bag limit to
five abalone/licence/day was recommended and accepted to assist with
recovery, and minimizing further risk of depletion of the abalone in key
recreational fishing regions along the East Coast.

In November 2021, the Tasmnaian Legislative Council rescinded the
five-bag limit regulation, with an immediate return to the 2018/2019
recreational abalone fishery regulations and limits. As part of the
motivation for the reversal of the bag limit reduction, it was argued
that size limits were the most important management lever available, and
that no efforts had been made to consider and/or implement increases in
the size limit governing the recreational abalone fishery.

Here we explore the consequences of increases in the recreational
abalone size limit on Tasmania's East Coast on the magnitude of the
recreational abalone fishery harvest.

\section{Estimate Recreational harvest for by mean
weight}\label{estimate-recreational-harvest-for-by-mean-weight}

Recreational abalone harvest has been routinely estimated by a
phone-diary survey involving a random sample of license-holders since
1996. Participants record details of their fishing activity including
date, location, method used, target species, start and finish times, and
the number of abalone kept (harvested) for the survey period 1st
November to 30th April the following year and thus only provide a
partial season estimate. Data from respondents are then expanded to
estimate the catch (number of abalone harvested) and effort of the
entire recreational licensed population using a `bootstraping' method to
estimate 95\% confidence limits using the percentile method. Estimated
harvest weight (kg) is then estimated by multiplying the harvest number
by the average weight of an individual abalone for each area based on
commercial catch sampling data collected prior to 2018 (area 1 = 522 g,
area 2 = 517 g, area 3 = 520 g) (Lyle et al.~2021).

The estimated combined east coast (areas 1-3) blacklip abalone
recreational harvest number for 2020-21 was n = 22882 (95\% CI
12774-34777) (area 1 = 11756, area 2 = 9919, area 3 = 1207) (Lyle et
al.~2021). In 2020-21 this equates to 11916 kg (11.9 t) (see Table 6 in
Lyle et al.~2021).

The estimated combined east coast (areas 1-3) blacklip abalone
recreational harvest number for 2018-19 was n = 25671 (95\% CI
13728-37139) (area 1 = 20339, area 2 = 2365, area 3 = 2967) (Lyle et
al.~2019). In 2018-19 this equates to 13167 kg (13.2 t) (see Table 6 in
Lyle et al.~2019).

The following code have been developed to utilise the most recent data
collected from the commercial abalone catch sampling program between
2019-2022.

\begin{figure}

{\centering \includegraphics[width=0.8\textwidth,height=0.8\textwidth]{MM_Length-Weight_RecreationalHarvest_files/figure-latex/lwplot check-1} 

}

\caption{Length-weight relationship of commercial abalone catch sampling data for all zones collected between 2019-2022.}\label{fig:lwplot check}
\end{figure}

\begin{figure}

{\centering \includegraphics[width=0.8\textwidth,height=0.8\textwidth]{MM_Length-Weight_RecreationalHarvest_files/figure-latex/east clean plot-1} 

}

\caption{Length-weight relationship of cleaned commercial abalone catch sampling data for the eastern zone collected between 2019-2022.}\label{fig:east clean plot}
\end{figure}

\begin{longtable}[]{@{}lrrr@{}}
\caption{Average weight of individual abalone from commerical abalone
catch sampling data for all zones collected between
2019-2022.}\tabularnewline
\toprule\noalign{}
Zone & Av.weight(g) & n & Catches \\
\midrule\noalign{}
\endfirsthead
\toprule\noalign{}
Zone & Av.weight(g) & n & Catches \\
\midrule\noalign{}
\endhead
\bottomrule\noalign{}
\endlastfoot
BS & 387.7818 & 889 & 9 \\
E & 562.3997 & 37438 & 386 \\
G & 662.9899 & 296 & 3 \\
N & 496.2601 & 1711 & 17 \\
W & 693.8298 & 27212 & 280 \\
\end{longtable}

\begin{longtable}[]{@{}rrrr@{}}
\caption{Average weight of individual abalone from commerical abalone
catch sampling data for all recreational harvest regions collected
between 2019-2022.}\tabularnewline
\toprule\noalign{}
RecRegion & Av.weight(g) & n & Catches \\
\midrule\noalign{}
\endfirsthead
\toprule\noalign{}
RecRegion & Av.weight(g) & n & Catches \\
\midrule\noalign{}
\endhead
\bottomrule\noalign{}
\endlastfoot
1 & 571.7411 & 39739 & 411 \\
3 & 519.6595 & 1580 & 16 \\
4 & 398.8881 & 563 & 6 \\
5 & 441.2958 & 649 & 6 \\
7 & 699.1372 & 1159 & 12 \\
8 & 688.3725 & 20078 & 206 \\
\end{longtable}

\section{Estimate recreational harvest weight for east coast based on
mean weight of individual abalone from commerical catch sampling for the
eastern
zone}\label{estimate-recreational-harvest-weight-for-east-coast-based-on-mean-weight-of-individual-abalone-from-commerical-catch-sampling-for-the-eastern-zone}

\begin{longtable}[]{@{}rrr@{}}
\caption{Estimated Recreational harvest based on average weight of
individual abalone from commerical catch sampling for the eastern zone
collected between 2019-2022.}\tabularnewline
\toprule\noalign{}
Av.weight(g) & Harvest 18/19(kg) & Harvest 20/21(kg) \\
\midrule\noalign{}
\endfirsthead
\toprule\noalign{}
Av.weight(g) & Harvest 18/19(kg) & Harvest 20/21(kg) \\
\midrule\noalign{}
\endhead
\bottomrule\noalign{}
\endlastfoot
562.3997 & 14437.36 & 12868.83 \\
\end{longtable}

\section{Estimate Recreational harvest based on length-weight
relationship}\label{estimate-recreational-harvest-based-on-length-weight-relationship}

\begin{longtable}[]{@{}lrrr@{}}
\caption{Estimated length-weight model parameters from commerical
abalone catch sampling data for all zones collected between
2019-2022.}\tabularnewline
\toprule\noalign{}
Zone & a & b & Length = 140 mm \\
\midrule\noalign{}
\endfirsthead
\toprule\noalign{}
Zone & a & b & Length = 140 mm \\
\midrule\noalign{}
\endhead
\bottomrule\noalign{}
\endlastfoot
BS & 0.0010634 & 2.617568 & 440.8935 \\
E & 0.0040640 & 2.363466 & 480.0188 \\
G & 0.0000692 & 3.162481 & 423.9968 \\
N & 0.0028971 & 2.433554 & 483.8185 \\
W & 0.0016766 & 2.551441 & 501.3590 \\
\end{longtable}

\begin{longtable}[]{@{}lrr@{}}
\caption{Estimated length-weight model parameters from commerical
abalone catch sampling data for all zones collected between
2019-2022.}\tabularnewline
\toprule\noalign{}
Zone & a & b \\
\midrule\noalign{}
\endfirsthead
\toprule\noalign{}
Zone & a & b \\
\midrule\noalign{}
\endhead
\bottomrule\noalign{}
\endlastfoot
BS & 0.0010634 & 2.617568 \\
E & 0.0040640 & 2.363466 \\
G & 0.0000692 & 3.162481 \\
N & 0.0028971 & 2.433554 \\
W & 0.0016766 & 2.551441 \\
\end{longtable}

\begin{longtable}[]{@{}
  >{\raggedleft\arraybackslash}p{(\columnwidth - 8\tabcolsep) * \real{0.0506}}
  >{\raggedleft\arraybackslash}p{(\columnwidth - 8\tabcolsep) * \real{0.2405}}
  >{\raggedleft\arraybackslash}p{(\columnwidth - 8\tabcolsep) * \real{0.2025}}
  >{\raggedleft\arraybackslash}p{(\columnwidth - 8\tabcolsep) * \real{0.2532}}
  >{\raggedleft\arraybackslash}p{(\columnwidth - 8\tabcolsep) * \real{0.2532}}@{}}
\caption{Estimated Recreational harvest for each LML based on estimated
harvest for 2020-21 season and abalone length-weight relationship from
commerical catch sampling for the eastern zone collected between
2019-2022.}\tabularnewline
\toprule\noalign{}
\begin{minipage}[b]{\linewidth}\raggedleft
LML
\end{minipage} & \begin{minipage}[b]{\linewidth}\raggedleft
Est. weight(g)
\end{minipage} & \begin{minipage}[b]{\linewidth}\raggedleft
Harvest(kg)
\end{minipage} & \begin{minipage}[b]{\linewidth}\raggedleft
Harvest 95\%(kg)
\end{minipage} & \begin{minipage}[b]{\linewidth}\raggedleft
Harvest 90\%(kg)
\end{minipage} \\
\midrule\noalign{}
\endfirsthead
\toprule\noalign{}
\begin{minipage}[b]{\linewidth}\raggedleft
LML
\end{minipage} & \begin{minipage}[b]{\linewidth}\raggedleft
Est. weight(g)
\end{minipage} & \begin{minipage}[b]{\linewidth}\raggedleft
Harvest(kg)
\end{minipage} & \begin{minipage}[b]{\linewidth}\raggedleft
Harvest 95\%(kg)
\end{minipage} & \begin{minipage}[b]{\linewidth}\raggedleft
Harvest 90\%(kg)
\end{minipage} \\
\midrule\noalign{}
\endhead
\bottomrule\noalign{}
\endlastfoot
138 & 463.9691 & 10616.54 & 10085.71 & 9554.886 \\
140 & 480.0188 & 10983.79 & 10434.60 & 9885.411 \\
145 & 521.5277 & 11933.60 & 11336.92 & 10740.237 \\
150 & 565.0349 & 12929.13 & 12282.67 & 11636.216 \\
155 & 610.5652 & 13970.95 & 13272.41 & 12573.858 \\
160 & 658.1429 & 15059.63 & 14306.64 & 13553.663 \\
\end{longtable}

\begin{longtable}[]{@{}
  >{\raggedleft\arraybackslash}p{(\columnwidth - 8\tabcolsep) * \real{0.0506}}
  >{\raggedleft\arraybackslash}p{(\columnwidth - 8\tabcolsep) * \real{0.2405}}
  >{\raggedleft\arraybackslash}p{(\columnwidth - 8\tabcolsep) * \real{0.2025}}
  >{\raggedleft\arraybackslash}p{(\columnwidth - 8\tabcolsep) * \real{0.2532}}
  >{\raggedleft\arraybackslash}p{(\columnwidth - 8\tabcolsep) * \real{0.2532}}@{}}
\caption{Estimated Recreational harvest for each LML based on estimated
harvest for 2018-19 season and abalone length-weight relationship from
commerical catch sampling for the eastern zone collected between
2019-2022.}\tabularnewline
\toprule\noalign{}
\begin{minipage}[b]{\linewidth}\raggedleft
LML
\end{minipage} & \begin{minipage}[b]{\linewidth}\raggedleft
Est. weight(g)
\end{minipage} & \begin{minipage}[b]{\linewidth}\raggedleft
Harvest(kg)
\end{minipage} & \begin{minipage}[b]{\linewidth}\raggedleft
Harvest 95\%(kg)
\end{minipage} & \begin{minipage}[b]{\linewidth}\raggedleft
Harvest 90\%(kg)
\end{minipage} \\
\midrule\noalign{}
\endfirsthead
\toprule\noalign{}
\begin{minipage}[b]{\linewidth}\raggedleft
LML
\end{minipage} & \begin{minipage}[b]{\linewidth}\raggedleft
Est. weight(g)
\end{minipage} & \begin{minipage}[b]{\linewidth}\raggedleft
Harvest(kg)
\end{minipage} & \begin{minipage}[b]{\linewidth}\raggedleft
Harvest 95\%(kg)
\end{minipage} & \begin{minipage}[b]{\linewidth}\raggedleft
Harvest 90\%(kg)
\end{minipage} \\
\midrule\noalign{}
\endhead
\bottomrule\noalign{}
\endlastfoot
138 & 463.9691 & 11910.55 & 11315.02 & 10719.50 \\
140 & 480.0188 & 12322.56 & 11706.43 & 11090.31 \\
145 & 521.5277 & 13388.14 & 12718.73 & 12049.32 \\
150 & 565.0349 & 14505.01 & 13779.76 & 13054.51 \\
155 & 610.5652 & 15673.82 & 14890.13 & 14106.44 \\
160 & 658.1429 & 16895.19 & 16050.43 & 15205.67 \\
\end{longtable}

\subsection{Determine relative change in Recreational harvest from
initial
LML}\label{determine-relative-change-in-recreational-harvest-from-initial-lml}

\begin{longtable}[]{@{}rrrr@{}}
\caption{Relative change in estimated Recreational harvest for an
increase in LML based on estimated harvest for 2020-21 season and
abalone length-weight relationship from commerical catch sampling for
the eastern zone collected between 2019-2022.}\tabularnewline
\toprule\noalign{}
LML & Est. weight (g) & Harvest (kg) & Harvest Change \\
\midrule\noalign{}
\endfirsthead
\toprule\noalign{}
LML & Est. weight (g) & Harvest (kg) & Harvest Change \\
\midrule\noalign{}
\endhead
\bottomrule\noalign{}
\endlastfoot
138 & 463.9691 & 10616.54 & 1.000000 \\
140 & 480.0188 & 10983.79 & 1.034592 \\
145 & 521.5277 & 11933.60 & 1.124057 \\
150 & 565.0349 & 12929.13 & 1.217829 \\
155 & 610.5652 & 13970.95 & 1.315961 \\
160 & 658.1429 & 15059.63 & 1.418506 \\
\end{longtable}

\begin{longtable}[]{@{}rrrr@{}}
\caption{Relative change in estimated Recreational harvest for an
increase in LML based on estimated harvest for 2018-19 season and
abalone length-weight relationship from commerical catch sampling for
the eastern zone collected between 2019-2022.}\tabularnewline
\toprule\noalign{}
LML & Est. weight (g) & Harvest (kg) & Harvest Change \\
\midrule\noalign{}
\endfirsthead
\toprule\noalign{}
LML & Est. weight (g) & Harvest (kg) & Harvest Change \\
\midrule\noalign{}
\endhead
\bottomrule\noalign{}
\endlastfoot
138 & 463.9691 & 11910.55 & 1.121886 \\
140 & 480.0188 & 12322.56 & 1.160695 \\
145 & 521.5277 & 13388.14 & 1.261064 \\
150 & 565.0349 & 14505.01 & 1.366265 \\
155 & 610.5652 & 15673.82 & 1.476358 \\
160 & 658.1429 & 16895.19 & 1.591402 \\
\end{longtable}

\begin{figure}

{\centering \includegraphics[width=0.8\textwidth,height=0.8\textwidth]{MM_Length-Weight_RecreationalHarvest_files/figure-latex/harvest relative change plot-1} \includegraphics[width=0.8\textwidth,height=0.8\textwidth]{MM_Length-Weight_RecreationalHarvest_files/figure-latex/harvest relative change plot-2} 

}

\caption{Estimated Recreational harvest in 2020-21 and 2018-19 for each LML based on abalone length-weight relationship from commerical catch sampling for the eastern zone collected between 2019-2021.}\label{fig:harvest relative change plot}
\end{figure}

\section{Estimate Recreational harvest based on commercial catch
sampling size
frequencies}\label{estimate-recreational-harvest-based-on-commercial-catch-sampling-size-frequencies}

\begin{longtable}[]{@{}rrrr@{}}
\caption{Estimated Recreational harvest for each LML based on estimated
harvest for 2020-21 season and abalone length-weight relationship and
size frequencies from commerical catch sampling for the eastern zone
collected between 2019-2022.}\tabularnewline
\toprule\noalign{}
LML & Est. Harvest(kg) & Harvest 95\%(kg) & Harvest 90\%(kg) \\
\midrule\noalign{}
\endfirsthead
\toprule\noalign{}
LML & Est. Harvest(kg) & Harvest 95\%(kg) & Harvest 90\%(kg) \\
\midrule\noalign{}
\endhead
\bottomrule\noalign{}
\endlastfoot
138 & 12830.72 & 12189.19 & 11547.65 \\
140 & 12987.04 & 12337.69 & 11688.33 \\
145 & 13677.40 & 12993.53 & 12309.66 \\
150 & 14504.68 & 13779.44 & 13054.21 \\
155 & 15412.46 & 14641.84 & 13871.21 \\
160 & 16380.69 & 15561.66 & 14742.62 \\
\end{longtable}

\begin{longtable}[]{@{}rrrr@{}}
\caption{Estimated Recreational harvest for each LML based on estimated
harvest for 2018-19 season and abalone length-weight relationship and
size frequencies from commerical catch sampling for the eastern zone
collected between 2019-2022.}\tabularnewline
\toprule\noalign{}
LML & Est. Harvest(kg) & Harvest 95\%(kg) & Harvest 90\%(kg) \\
\midrule\noalign{}
\endfirsthead
\toprule\noalign{}
LML & Est. Harvest(kg) & Harvest 95\%(kg) & Harvest 90\%(kg) \\
\midrule\noalign{}
\endhead
\bottomrule\noalign{}
\endlastfoot
138 & 14394.61 & 13674.88 & 12955.15 \\
140 & 14569.98 & 13841.48 & 13112.98 \\
145 & 15344.48 & 14577.26 & 13810.03 \\
150 & 16272.60 & 15458.97 & 14645.34 \\
155 & 17291.02 & 16426.47 & 15561.92 \\
160 & 18377.27 & 17458.41 & 16539.55 \\
\end{longtable}

\begin{longtable}[]{@{}rrr@{}}
\caption{Relative change in estimated Recreational harvest for an
increase in LML based on estimated harvest for 2020-21 season and
abalone length-weight relationship and size frequencies from commerical
catch sampling for the eastern zone collected between
2019-2022.}\tabularnewline
\toprule\noalign{}
LML & Harvest(kg) & HarvestChange \\
\midrule\noalign{}
\endfirsthead
\toprule\noalign{}
LML & Harvest(kg) & HarvestChange \\
\midrule\noalign{}
\endhead
\bottomrule\noalign{}
\endlastfoot
138 & 12830.72 & 1.000000 \\
140 & 12987.04 & 1.012183 \\
145 & 13677.40 & 1.065988 \\
150 & 14504.68 & 1.130465 \\
155 & 15412.46 & 1.201215 \\
160 & 16380.69 & 1.276677 \\
\end{longtable}

\begin{longtable}[]{@{}rrr@{}}
\caption{Relative change in estimated Recreational harvest for an
increase in LML based on estimated harvest for 2018-19 season and
abalone length-weight relationship and size frequencies from commerical
catch sampling for the eastern zone collected between
2019-2022.}\tabularnewline
\toprule\noalign{}
LML & Harvest(kg) & HarvestChange \\
\midrule\noalign{}
\endfirsthead
\toprule\noalign{}
LML & Harvest(kg) & HarvestChange \\
\midrule\noalign{}
\endhead
\bottomrule\noalign{}
\endlastfoot
138 & 14394.61 & 1.000000 \\
140 & 14569.98 & 1.012183 \\
145 & 15344.48 & 1.065988 \\
150 & 16272.60 & 1.130465 \\
155 & 17291.02 & 1.201215 \\
160 & 18377.27 & 1.276677 \\
\end{longtable}

\begin{figure}

{\centering \includegraphics[width=0.8\textwidth,height=0.8\textwidth]{MM_Length-Weight_RecreationalHarvest_files/figure-latex/lw harvest relative change plot-1} \includegraphics[width=0.8\textwidth,height=0.8\textwidth]{MM_Length-Weight_RecreationalHarvest_files/figure-latex/lw harvest relative change plot-2} 

}

\caption{Estimated Recreational harvest in 2020-21 and 2018-19 for each LML based on abalone length-weight relationship and size frequencies from commerical catch sampling for the eastern zone collected between 2019-2022.}\label{fig:lw harvest relative change plot}
\end{figure}

\section{References}\label{references}

Lyle, J.M., Ewing, F., Ewing, G. and Tracey, S.R. (2021). Tasmanian
recreational Rock Lobster and Abalone fisheries: 2020-21 fishing season.
Institute for Marine and Antarctic Studies Report, 38p.

Lyle, J.M., Ewing, F., Ewing, G. and Tracey, S.R. (2019). Tasmanian
recreational Rock Lobster and Abalone fisheries: 2018-19 fishing season.
Institute for Marine and Antarctic Studies Report, 36p.

\end{document}
