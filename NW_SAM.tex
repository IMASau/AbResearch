% Options for packages loaded elsewhere
\PassOptionsToPackage{unicode}{hyperref}
\PassOptionsToPackage{hyphens}{url}
\PassOptionsToPackage{dvipsnames,svgnames,x11names}{xcolor}
%
\documentclass[
  A4paperpaper,
  DIV=11,
  numbers=noendperiod]{scrreprt}

\usepackage{amsmath,amssymb}
\usepackage{iftex}
\ifPDFTeX
  \usepackage[T1]{fontenc}
  \usepackage[utf8]{inputenc}
  \usepackage{textcomp} % provide euro and other symbols
\else % if luatex or xetex
  \usepackage{unicode-math}
  \defaultfontfeatures{Scale=MatchLowercase}
  \defaultfontfeatures[\rmfamily]{Ligatures=TeX,Scale=1}
\fi
\usepackage{lmodern}
\ifPDFTeX\else  
    % xetex/luatex font selection
  \setmainfont[]{Arial}
\fi
% Use upquote if available, for straight quotes in verbatim environments
\IfFileExists{upquote.sty}{\usepackage{upquote}}{}
\IfFileExists{microtype.sty}{% use microtype if available
  \usepackage[]{microtype}
  \UseMicrotypeSet[protrusion]{basicmath} % disable protrusion for tt fonts
}{}
\makeatletter
\@ifundefined{KOMAClassName}{% if non-KOMA class
  \IfFileExists{parskip.sty}{%
    \usepackage{parskip}
  }{% else
    \setlength{\parindent}{0pt}
    \setlength{\parskip}{6pt plus 2pt minus 1pt}}
}{% if KOMA class
  \KOMAoptions{parskip=half}}
\makeatother
\usepackage{xcolor}
\usepackage[left = 20mm,right = 20mm,top = 20mm,bottom = 10mm]{geometry}
\setlength{\emergencystretch}{3em} % prevent overfull lines
\setcounter{secnumdepth}{4}
% Make \paragraph and \subparagraph free-standing
\ifx\paragraph\undefined\else
  \let\oldparagraph\paragraph
  \renewcommand{\paragraph}[1]{\oldparagraph{#1}\mbox{}}
\fi
\ifx\subparagraph\undefined\else
  \let\oldsubparagraph\subparagraph
  \renewcommand{\subparagraph}[1]{\oldsubparagraph{#1}\mbox{}}
\fi


\providecommand{\tightlist}{%
  \setlength{\itemsep}{0pt}\setlength{\parskip}{0pt}}\usepackage{longtable,booktabs,array}
\usepackage{calc} % for calculating minipage widths
% Correct order of tables after \paragraph or \subparagraph
\usepackage{etoolbox}
\makeatletter
\patchcmd\longtable{\par}{\if@noskipsec\mbox{}\fi\par}{}{}
\makeatother
% Allow footnotes in longtable head/foot
\IfFileExists{footnotehyper.sty}{\usepackage{footnotehyper}}{\usepackage{footnote}}
\makesavenoteenv{longtable}
\usepackage{graphicx}
\makeatletter
\def\maxwidth{\ifdim\Gin@nat@width>\linewidth\linewidth\else\Gin@nat@width\fi}
\def\maxheight{\ifdim\Gin@nat@height>\textheight\textheight\else\Gin@nat@height\fi}
\makeatother
% Scale images if necessary, so that they will not overflow the page
% margins by default, and it is still possible to overwrite the defaults
% using explicit options in \includegraphics[width, height, ...]{}
\setkeys{Gin}{width=\maxwidth,height=\maxheight,keepaspectratio}
% Set default figure placement to htbp
\makeatletter
\def\fps@figure{htbp}
\makeatother

\KOMAoption{captions}{tableheading}
\usepackage{float}
\floatplacement{table}{H}
\makeatletter
\@ifpackageloaded{caption}{}{\usepackage{caption}}
\AtBeginDocument{%
\ifdefined\contentsname
  \renewcommand*\contentsname{Table of contents}
\else
  \newcommand\contentsname{Table of contents}
\fi
\ifdefined\listfigurename
  \renewcommand*\listfigurename{List of Figures}
\else
  \newcommand\listfigurename{List of Figures}
\fi
\ifdefined\listtablename
  \renewcommand*\listtablename{List of Tables}
\else
  \newcommand\listtablename{List of Tables}
\fi
\ifdefined\figurename
  \renewcommand*\figurename{Figure}
\else
  \newcommand\figurename{Figure}
\fi
\ifdefined\tablename
  \renewcommand*\tablename{Table}
\else
  \newcommand\tablename{Table}
\fi
}
\@ifpackageloaded{float}{}{\usepackage{float}}
\floatstyle{ruled}
\@ifundefined{c@chapter}{\newfloat{codelisting}{h}{lop}}{\newfloat{codelisting}{h}{lop}[chapter]}
\floatname{codelisting}{Listing}
\newcommand*\listoflistings{\listof{codelisting}{List of Listings}}
\makeatother
\makeatletter
\makeatother
\makeatletter
\@ifpackageloaded{caption}{}{\usepackage{caption}}
\@ifpackageloaded{subcaption}{}{\usepackage{subcaption}}
\makeatother
\ifLuaTeX
  \usepackage{selnolig}  % disable illegal ligatures
\fi
\usepackage{bookmark}

\IfFileExists{xurl.sty}{\usepackage{xurl}}{} % add URL line breaks if available
\urlstyle{same} % disable monospaced font for URLs
\hypersetup{
  pdftitle={Assessing growth dynamics and connectivity of blacklip abalone (Haliotis rubra) populations in NW Tasmania},
  pdfauthor={Jaime McAllister},
  colorlinks=true,
  linkcolor={blue},
  filecolor={Maroon},
  citecolor={Blue},
  urlcolor={Blue},
  pdfcreator={LaTeX via pandoc}}

\title{Assessing growth dynamics and connectivity of blacklip abalone
(Haliotis rubra) populations in NW Tasmania}
\usepackage{etoolbox}
\makeatletter
\providecommand{\subtitle}[1]{% add subtitle to \maketitle
  \apptocmd{\@title}{\par {\large #1 \par}}{}{}
}
\makeatother
\subtitle{AOTF Progress Report: Size-at-maturity (SAM)}
\author{Jaime McAllister}
\date{Last Updated on 14 November, 2025}

\begin{document}
\maketitle

\chapter{Overview}\label{overview}

This milestone report provides an overview of progress to date in
collecting new data to assess the reproductive condition of blacklip
abalone in North West Tasmania, as part of the Abalone Industry
Development Fund (AIDF) project Assessing Growth Dynamics and
Connectivity of Blacklip Abalone (Haliotis rubra) Populations in NW
Tasmania.

\chapter{Methods}\label{methods}

\section{Abalone Collection and Handling
Procedures}\label{abalone-collection-and-handling-procedures}

As of 14 November 2025, abalone have been collected from 12 sites within
Block 5 (see Figure~\ref{fig-map}), with support from industry
contractors who provided charter vessels and contributed local knowledge
and logistical expertise.

At each site, 200 abalone were sampled by targeting between 20--40
individuals within each 20 mm size class increment from 60 mm to 140 mm.
Collected abalone were placed in plastic fish bins, covered with hessian
sacks, and kept moist and exposed to air until landing.

Upon landing, abalone were transferred to a refrigerated vehicle and
transported to Tasmanian Seafoods. There, they were either held
overnight in a refrigerated room (4\,°C) or in live holding tanks prior
to processing the following day.

During processing, abalone were removed from their shells and assessed
for maturity status was determined following Jones et al.~2009:

\begin{itemize}
\tightlist
\item
  Stage 0, has no apparent development of gonad (immature).
\item
  Stage 1, gonad development has started, such that it is possible to
  determine sex of animal, although the gonad at this stage is very
  slight, at its most developed form it is translucent so that the
  digestive gland is still visible underneath (immature).
\item
  Stage 2, gonad is obvious at the extremities of the digestive gland,
  it is opaque but not yet fully formed. The eggs in females are visible
  at low magnification while males are viscous creamy yellow (mature).
\item
  Stage 3, fully formed gonad (mature). Stages 1 to 3 can be grouped by
  sex but only stages 2 and 3 are considered mature as although in stage
  1 sex may be determined, that individual is unlikely to be
  reproductive and so is categorised as immature male or female
  (mature).
\end{itemize}

\begin{figure}

\centering{

\includegraphics{NW_SAM_files/figure-pdf/fig-map-1.pdf}

}

\caption{\label{fig-map}Map of size-at-maturity sites sampled in North
West Tasmania (Block 5) as of 14 November 2025.}

\end{figure}%

\section{Data analysis}\label{data-analysis}

Size at maturity (SAM) estimation and analysis has been performed using
the `biology' package in R developed by Malcom Haddon (Haddon 2025).

A dataframe has been created to run the `fitmaturity' function where
maturity has been classified as:

\begin{itemize}
\tightlist
\item
  I = stages 0-1
\item
  M = stages 2-3.
\end{itemize}

\chapter{Results}\label{results}

\section{Data checks and summaries}\label{data-checks-and-summaries}

Quick summary plots to look for outliers in length data. Summary counts
for each of the target size classes from dive collections.

An initial examination of the raw data revealed no obvious outliers that
warranted exclusion from further analysis.

\includegraphics{NW_SAM_files/figure-pdf/datachecks-1.pdf}

\includegraphics{NW_SAM_files/figure-pdf/datachecks-2.pdf}

\includegraphics{NW_SAM_files/figure-pdf/datachecks-3.pdf}

\includegraphics{NW_SAM_files/figure-pdf/datachecks-4.pdf}

\includegraphics{NW_SAM_files/figure-pdf/datachecks-5.pdf}

\section{Size-at-maturity by site}\label{size-at-maturity-by-site}

There was a broad range in size-at-maturity across the 12 sampled sites,
ranging from 96.6 mm at Site 10 to 122.2 mm at Site 6 (Figures 3.1 to
3.3). This variation is consistent with our understanding of the spatial
complexity and distinct population characteristics of abalone in this
region of the fishery.

\begin{figure}

\centering{

\includegraphics{NW_SAM_files/figure-pdf/fig-prop-mature-1.pdf}

}

\caption{\label{fig-prop-mature-1}Proportion mature at length for
blacklip abalone maturity data at each surveyed site in Block 5 (NW
Tasmania) in 2025. Length at 50\% maturity is indicated by the green
vertical line, and sample size is presented in the top-left corner of
the plot.}

\end{figure}%

\begin{figure}

\centering{

\includegraphics{NW_SAM_files/figure-pdf/fig-prop-mature-2.pdf}

}

\caption{\label{fig-prop-mature-2}Proportion mature at length for
blacklip abalone maturity data at each surveyed site in Block 5 (NW
Tasmania) in 2025. Length at 50\% maturity is indicated by the green
vertical line, and sample size is presented in the top-left corner of
the plot.}

\end{figure}%

\begin{figure}

\centering{

\includegraphics{NW_SAM_files/figure-pdf/fig-prop-mature-3.pdf}

}

\caption{\label{fig-prop-mature-3}Proportion mature at length for
blacklip abalone maturity data at each surveyed site in Block 5 (NW
Tasmania) in 2025. Length at 50\% maturity is indicated by the green
vertical line, and sample size is presented in the top-left corner of
the plot.}

\end{figure}%

\section{Size-at-maturity across Block
5}\label{size-at-maturity-across-block-5}

Preliminary analysis combining all sites sampled to data suggests that
the overall size-at-maturity for Block 5 is approximately 109.3 mm
(Figure~\ref{fig-prop-mature-block}).

\begin{figure}

\centering{

\includegraphics{NW_SAM_files/figure-pdf/fig-prop-mature-block-1.pdf}

}

\caption{\label{fig-prop-mature-block}Proportion mature at length for
blacklip abalone maturity data for Block 5 (NW Tasmania) in 2025. Length
at 50\% maturity is indicated by the green vertical line, and sample
size is presented in the top-left corner of the plot.}

\end{figure}%

\chapter{Shellology maturity status}\label{shellology-maturity-status}

The following analyses were conducted as part of the collaborative AIDF
project, `Assessing the Potential of SPR Methods to Improve Fishery
Assessments and Management Decisions', sub-contracted to Dr Jeremy
Prince.

\section{Internal shell scaring vs
maturity}\label{internal-shell-scaring-vs-maturity}

A quick summary plot comparing Prince `shellology' classification of
internal shell scaring and maturity status determined from macroscopic
examination of gonads. A summary of the Shell\_internal\_score:

\begin{enumerate}
\def\labelenumi{\arabic{enumi}.}
\tightlist
\item
  No scar formation.
\item
  Some pitting.
\item
  Quite pitted.
\item
  Secondary deposition forming.
\item
  Secondary deposition covers most of muscle attachment site.
\item
  Secondary deposition covers all of muscle attachment site.
\end{enumerate}

\begin{figure}

\centering{

\includegraphics{NW_SAM_files/figure-pdf/fig-int-shell-count-1.pdf}

}

\caption{\label{fig-int-shell-count}Total count of abalone for each
internal shell classification from specimens sampled in Block 5 (NW
Tasmania) in 2025.}

\end{figure}%

Preliminary observations of internal shell condition suggest that
scarring and secondary deposition become increasingly evident with
maturation, supporting its potential use as an indicator of reproductive
status (Figure~\ref{fig-int-shell-count}). However, it is important to
note that this remains a destructive sampling method, requiring animals
to be removed from their shells to assess internal shell features.

\section{External shell appearance vs
maturity}\label{external-shell-appearance-vs-maturity}

A quick summary plot comparing Prince `shellology' classification of
external shell appearance and maturity status determined from external
appearance and fouling of shell. A summary of the
Shell\_external\_score:

\begin{enumerate}
\def\labelenumi{\arabic{enumi}.}
\tightlist
\item
  Clean shell; no epiphytic growth.
\item
  Some epiphytic fouling but not advanced.
\item
  Hard fouling commenced around spire.
\item
  Fleshy fouling and coraline covering spire but ripple texture of shell
  apparent.
\item
  Fleshy fouling and coraline covering most of shell and ripple texture
  of shell non-apparent
\item
  Shell completely overgrown with hard fouling to growing edge.
\item
  Very thick fouling and shell completely overgrown.
\end{enumerate}

\begin{figure}

\centering{

\includegraphics{NW_SAM_files/figure-pdf/fig-ext-shell-count-1.pdf}

}

\caption{\label{fig-ext-shell-count}Total count of abalone for each
external shell classification from specimens sampled in Block 5 (NW
Tasmania) in 2025.}

\end{figure}%

Preliminary observations align with our understanding of emergence
patterns, with immature juvenile abalone typically exhibiting clean
external shells and minimal coralline growth or other fouling
(Figure~\ref{fig-ext-shell-count}). As animals mature, external fouling
becomes increasingly apparent, supporting its potential use as a
non-destructive macroscopic indicator of reproductive condition.

\chapter{Summary}\label{summary}

Preliminary analysis of data from the 12 sampled sites revealed clear
variation in size-at-maturity across the block, consistent with our
initial understanding of abalone population dynamics in the region. The
inclusion of additional shell classification metrics has provided a
complementary means of validating reproductive condition, which will be
further explored as the project progresses.

This work has now been incorporated into the broader AIRF 2023-64
project, which aims to examine abalone population dynamics across the
region. The expanded scope includes additional size-at-maturity
sampling, tagging and growth experiments, genetic and physiological
assessments, and continued development of shell condition metrics to
support SPR-based assessment approaches.

Estimates of legal minimum length, defined as size-at-maturity plus
three years of post-maturity growth, have not yet been updated, pending
further data from ongoing tagging studies. These updates will form part
of the broader project, which is currently underway and scheduled for
completion by late 2026.



\end{document}
